\documentclass[11pt]{standalone}	% Required - Tells LaTeX how to display the document.

\title{Standalone}
\date{May 30th, 2018}
\author{Daniel Borrus}

%% Preamble %%

	\usepackage{inputenc}	% Prepares LaTeX engine for non-ascii characters, such as UTF-8 characters. Important.

	% Page Layout
	%\usepackage{geometry} % set page width
	%\usepackage[margin=1cm]{caption} % set caption width
	%\usepackage{float} % Needed for creating figure “floats”

	% Page Display
	\usepackage{url} % Allows for url links
	\usepackage{graphicx} % Allows for inputing figures
	\usepackage[dvipsnames]{xcolor} % For pretty colors

	% Math Typesetting
	\usepackage{amsmath} % Provides useful equation formats
	\usepackage{amssymb} % Provieds a ton of useful math symbols
	\usepackage{mathtools} % Provides patches for amsmath

	% TikZ/pgfPlots
	\usepackage{tikz}
	\usepackage{pgfplots}
	%\pgfplotsset{compat=1.14}
	\usetikzlibrary{automata,arrows,positioning,calc}
	\pgfplotsset{ every non boxed x axis/.append style={x axis line style=-}, every non boxed y axis/.append style={y axis line style=-}}


\begin{document}

	\centering
	\begin{tikzpicture} 
	
		\node at (3,11.5) {\textbf{\textit{in vitro}}};
	
		\begin{axis}[
		height = 3 cm,
		width  = 8 cm,
		yshift = 8.5cm,
		axis x line = none,
		axis y line = left,
		xtick = {0},
		ytick = {0},
		]
		\addplot[] {sin(deg(x))}; 
		\end{axis}
		
		\begin{axis}[
		height = 3 cm,
		width  = 8 cm,
		yshift = 5.5cm,
		axis x line = none,
		axis y line = left,
		xtick = {0},
		ytick = {0},			
		]
		\addplot[] {sin(deg(x))}; 
		\end{axis}
	
		\begin{axis}[
			height= 6cm,
			width = 8cm,
			axis x line = bottom,
			axis y line = left,
			ytick= {0,0.1,0.2,0.4},
			xmax=9.5,xmin=2.5,ymin=0,ymax=0.25,
			ylabel = {eupnea\\sec$^{(-1)}$},
			y label style = {rotate=-90,yshift=0.4cm,xshift=0.3cm,Blue,align=left},
			xlabel = {$[K^+]_{\text{Out}}$},	
			x dir = reverse,	
			]
			\foreach \x in {0,...,6}
			{
				\pgfmathsetmacro{\y}{int(\x+7)}
				\addplot[only marks,blue,draw=black,opacity=0.25,mark size=2] table[x index=\x ,y index= \y] {../data/PKdata_eup.dat};
			}
			\addplot[blue,only marks,mark=diamond*,draw=black,mark size=4] table[y index=3,x index=1] {../data/PKdata_means.dat};
		\end{axis}
		\begin{axis}[
		height= 6cm,
		width = 8cm,
		axis x line = bottom,
		axis y line = right,
		ytick= {0,0.5,1},
		xmax=9.5,xmin=2.5,
		ymin=0,ymax=1.1,
		ylabel = sighs\\min$^{(-1)}$,
		y label style = {rotate=-90,yshift=-1.2cm,xshift=-0.3cm,Red,align=left},
		xlabel = {$[K^+]_{\text{Out}}$},	
		x dir = reverse,	
		]
		\foreach \x in {0,...,6}
		{
			\pgfmathsetmacro{\y}{int(\x+7)}
			\addplot[only marks,red,draw=black,opacity=0.25, mark size = 2] table[x index=\x ,y index= \y] {../data/PKdata_sig.dat};
		}
		\addplot[red,only marks,mark=diamond*,draw=black,mark size=4] table[y index=2,x index=0] {../data/PKdata_means.dat};
		\addplot[red,thick] table {../data/PKdata_sigfit.dat};
		\end{axis}
		% just the blue line
		\begin{axis}[
		height= 6cm,
		width = 8cm,
		axis x line = bottom,
		axis y line = left,
		ytick= {-1},
		xmax=9.5,xmin=2.5,ymin=0,ymax=0.25,
		x dir = reverse,	
		]
		\addplot[blue,thick] table {../data/PKdata_eupfit.dat};
		\end{axis}
		\begin{axis}[
		height= 6cm,
		width = 8cm,
		axis x line = bottom,
		axis y line = right,
		ytick= {-1},
		xmax=9.5,xmin=2.5,
		ymin=0,ymax=1.1,	
		x dir = reverse,	
		]
		\addplot[red,thick] table {../data/PKdata_sigfit.dat};
		\end{axis}
%		\begin{axis}[
%		height= 6cm,
%		width = 8cm,
%		yshift=-6cm,
%		axis x line = bottom,
%		axis y line = left,
%		ytick= {0,0.2,0.4},
%		xmax=9.5,xmin=2.5,ymin=0,ymax=0.25,
%		ylabel = {eupnea\\sec$^{(-1)}$},
%		y label style = {rotate=-90,yshift=0.8cm,xshift=-0.2cm,Blue,align=left},
%		xlabel = {$[K^+]_{\text{Out}}$},	
%		x dir = reverse,	
%		]
%		\addplot[blue,dashed,mark=diamond*,draw=black,mark size=3] table[y index=2] {../data/PKdata_means.dat};
%		%			\foreach \x in {0,...,6}
%		%			{
%		%				\pgfmathsetmacro{\y}{int(\x+7)}
%		%				\addplot[only marks,blue,draw=black,opacity=0.5,mark size=2] table[x index=\x ,y index= \y] {../data/PKdata_eup.dat};
%		%			}
%		\end{axis}
		
	\end{tikzpicture}

\end{document}