\documentclass[11pt,margin=1cm,varwidth]{standalone}	% Required - Tells LaTeX how to display the document.

\title{Standalone}
\date{May 4th, 2020}
\author{Daniel Borrus}

%% Preamble %%

\usepackage{inputenc}	% Prepares LaTeX engine for non-ascii characters, such as UTF-8 characters. Important.

% Page Layout
%\usepackage[margin=0in,tmargin=.8in]{geometry} % set page width
%\usepackage[margin=1cm]{caption} % set caption width
\usepackage{float} % Needed for creating figure “floats”

% Page Display
\usepackage{url} % Allows for url links
\usepackage{graphicx} % Allows for inputing figures
\usepackage[dvipsnames]{xcolor} % For pretty colors

% Math Typesetting
\usepackage{amsmath} % Provides useful equation formats
\usepackage{amssymb} % Provieds a ton of useful math symbols
\usepackage{mathtools} % Provides patches for amsmath

% TikZ/pgfPlots
\usepackage{tikz}
\usepackage{pgfplots}
\pgfplotsset{compat=1.14}
\usetikzlibrary{automata,arrows,positioning,calc}
\pgfplotsset{ every non boxed x axis/.append style={x axis line style=-}, every non boxed y axis/.append style={y axis line style=-}}

% Miscellaneous
\usepackage{soul}
\newcommand{\highlight}[1]{%
	\colorbox{yellow!50}{$\displaystyle#1$}}

% Unique for this document

%\input{mydef.tex}

\begin{document}	

	{ \bf Closed Ca$^{2+}$ sub-system} \\
	Increasing peak ER Ca$^{2+}$ efflux via simulated application of NMB 

	\begin{center}
	
	\begin{align*}
	\frac{dc}{dt} &= \left[ \highlight{v_1 f_\infty(c)} + v_2 \right] \left[ c_{er} - c \right] - \frac{v_3 c^2}{k^2_3+c^2} + j_0 - \frac{v_4c^4}{k^4_4 + c^4}\\
	\frac{dc_{tot}}{dt} &= j_0 - \frac{v_4c^4}{k^4_4 + c^4}
	\end{align*}
	
	\includegraphics[width=5in]{../../finf.png}
	
	\includegraphics[width=5in]{../../trajectories.png}



	\end{center}

	Doubling v1 (peak Ca${^2+}$) doubles sigh rhythm frequency
	
	\vskip10mm
	
	{ \bf What is $v_1$ changing in the model?} \\
	Observing the nulc lines in the phase diagram. 
	
	\begin{center}
		\includegraphics[width=3in]{../../phasediagram.png}
	\end{center}
	Observing the bifurcation diagram with $v_1$. Note, we needed to scale cyt. Ca dynamics by 1/10.
	\begin{center}
		\includegraphics[width=4in]{../../MatCont7p2/MatCont7p2/take2Figure.png}
	
	\end{center}
	
\end{document}